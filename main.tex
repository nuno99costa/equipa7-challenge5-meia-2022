\documentclass[conference]{IEEEtran}

\usepackage{cite}
\usepackage{amsmath,amssymb,amsfonts}
\usepackage{algorithmic}
\usepackage{graphicx}
\usepackage{textcomp}
\usepackage{xcolor}
\usepackage{tikz}
\usepackage{pgfplots}
\pgfplotsset{compat=1.18}
\usepackage[acronym]{glossaries}

\begin{document}

\title{A systematic review on produce defect detection and quality assessement using artificial intelligence}

\author{
    \IEEEauthorblockN{1\textsuperscript{st} João Silva}
    \IEEEauthorblockA{\textit{Departamento de Engenharia Informática} \\
    \textit{Instituto de Superior de Engenharia do Porto}\\
    1150425@isep.ipp.pt}
    \\
    \IEEEauthorblockN{3\textsuperscript{th} Nuno Costa}
    \IEEEauthorblockA{\textit{Departamento de Engenharia Informática} \\
    \textit{Instituto de Superior de Engenharia do Porto}\\
    1171584@isep.ipp.pt}
    \\
    \IEEEauthorblockN{5\textsuperscript{th} Diogo Formosinho}
    \IEEEauthorblockA{\textit{Departamento de Engenharia Informática} \\
    \textit{Instituto de Superior de Engenharia do Porto}\\
    1210056@isep.ipp.pt}
    \and
    \IEEEauthorblockN{2\textsuperscript{nd} João Santos}
    \IEEEauthorblockA{\textit{Departamento de Engenharia Informática} \\
    \textit{Instituto de Superior de Engenharia do Porto}\\
    1161023@isep.ipp.pt}
    \\
    \IEEEauthorblockN{4\textsuperscript{th} José Araújo}
    \IEEEauthorblockA{\textit{Departamento de Engenharia Informática} \\
    \textit{Instituto de Superior de Engenharia do Porto}\\
    1180943@isep.ipp.pt}
    \\
    \IEEEauthorblockN{6\textsuperscript{th} Francisco Cabrita}
    \IEEEauthorblockA{\textit{Departamento de Engenharia Informática} \\
    \textit{Instituto de Superior de Engenharia do Porto}\\
    1210058@isep.ipp.pt}
}
\maketitle

\begin{abstract}

\end{abstract}

\begin{IEEEkeywords}
\end{IEEEkeywords}

\section{Introduction}

Agriculture is one of the most important economic activities, sustaining livelihoods by securing food production and providing income \cite{FDES-1}. This sector has and continues to undergo a process of steady industrialization, through the increase of commercial farm size, commodity specialization and increase capital availability, among other factors \cite{10.2307/1243439}. This industrialization has caused an explosive increase in productivity for nearly all agricultural activity \cite{owidagriculturalproduction}. In order to take full advantage of this increased productivity, improvements are also required in the infrastructure and post harvest processes of a given agricultural unit, before the produce reaches the final consumer \cite{Food_and_Agriculture_Organization_of_the_United_Nations2010-hb}.

\section{Method}

\subsection{Research Questions}

\subsection{Data Sources}

\subsection{Search Terms}

\subsection{Quality assessment: inclusion and exclusion criteria}

\subsection{Data Extraction}

\section{Results}

\subsection{RQX}

\section{Discussion}

\subsection{RQX}

\section{Conclusion and future work}

\bibliographystyle{IEEEtran}
\bibliography{refs}

\end{document}