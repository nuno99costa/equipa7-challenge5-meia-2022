\documentclass[conference]{IEEEtran}

\usepackage{cite}
\usepackage{amsmath,amssymb,amsfonts}
\usepackage{algorithmic}
\usepackage{graphicx}
\usepackage{textcomp}
\usepackage{xcolor}
\usepackage{tikz}
\usepackage{pgfplots}
\pgfplotsset{compat=1.18}
\usepackage[acronym]{glossaries}

\begin{document}

\title{A systematic review on produce defect detection and quality assessement using artificial intelligence}

\author{
    \IEEEauthorblockN{1\textsuperscript{st} João Silva}
    \IEEEauthorblockA{\textit{Departamento de Engenharia Informática} \\
    \textit{Instituto de Superior de Engenharia do Porto}\\
    1150425@isep.ipp.pt}
    \\
    \IEEEauthorblockN{3\textsuperscript{th} Nuno Costa}
    \IEEEauthorblockA{\textit{Departamento de Engenharia Informática} \\
    \textit{Instituto de Superior de Engenharia do Porto}\\
    1171584@isep.ipp.pt}
    \\
    \IEEEauthorblockN{5\textsuperscript{th} Diogo Formosinho}
    \IEEEauthorblockA{\textit{Departamento de Engenharia Informática} \\
    \textit{Instituto de Superior de Engenharia do Porto}\\
    1210056@isep.ipp.pt}
    \and
    \IEEEauthorblockN{2\textsuperscript{nd} João Santos}
    \IEEEauthorblockA{\textit{Departamento de Engenharia Informática} \\
    \textit{Instituto de Superior de Engenharia do Porto}\\
    1161023@isep.ipp.pt}
    \\
    \IEEEauthorblockN{4\textsuperscript{th} José Araújo}
    \IEEEauthorblockA{\textit{Departamento de Engenharia Informática} \\
    \textit{Instituto de Superior de Engenharia do Porto}\\
    1180943@isep.ipp.pt}
    \\
    \IEEEauthorblockN{6\textsuperscript{th} Francisco Cabrita}
    \IEEEauthorblockA{\textit{Departamento de Engenharia Informática} \\
    \textit{Instituto de Superior de Engenharia do Porto}\\
    1210058@isep.ipp.pt}
}
\maketitle

\begin{abstract}

\end{abstract}

\begin{IEEEkeywords}
\end{IEEEkeywords}

\section{Introduction}

Agriculture is one of the most important economic activities, sustaining livelihoods by securing food production and providing income \cite{FDES-1}, with a global gross value of 4.145 trillion US dollars in 2020 \cite{FAO1}. This sector has and continues to undergo a process of steady industrialization, through the increase of commercial farm size, commodity specialization and increase capital availability, among other factors \cite{10.2307/1243439}. This industrialization has caused an explosive increase in productivity for nearly all agricultural activity \cite{owidagriculturalproduction}. In order to take full advantage of this increased productivity, improvements are also required in the infrastructure and post harvest processes of a given agricultural unit, before the produce reaches the final consumer \cite{Food_and_Agriculture_Organization_of_the_United_Nations2010-hb}. With this, the post-harvest process has become crucial to reduce waste, preserve said foods and provide the end consumer with fresh, quality items. The previously mentioned improvements have materialized into facilities in which the produce can be stored, cleaned, prepared, sorted and packaged. However, with the large throughput of produce being processed in the aforementioned facilities, there is a need of system to effectively monitor and sort said produce, taking into account defects and overall quality.

Furthermore, markets are usually regulated to include minimum requirements on produce quality and appearance, such as European Commission Implementing Regulation (EU) 543/2011, which sets out minimum quality requirements for a variety of produce, such as intactness, soundness and cleanliness \cite{eu-5432011}. In order to abide by the aforementioned regulation, producers must consider travel time and the possibility of spoilage within that time frame. Small defects in produce expand over time and can potentially spoil the whole item (be it a single produce item or a packaged set of items). Detection systems are already in place in post harvesting processes in the horticultural sector, identifying pathological, mechanical, physiological and internal defects \cite{Nturambirwe2020}.

The purpose of this work is to understand what factors can be monitored to ascertain the quality or defect presence in produce, what techniques and technologies are available to analyse and monitor produce in a industrial environment and how is artificial intelligence and related fields enhancing the usage of said technologies. To do this, \textcolor{red}{X} research questions are presented, and existing studies that may answer these questions are analysed and discussed in order to reach an overview of the current state of the art in the topics of produce quality analysis and defect detection using artificial intelligence, the evolution of the domain and their application in the post-harvest processing field.

The remainder of this document is structured in 4 sections:
\begin{itemize}
	\item Section \ref{sec:meth}, in which the procedure taken in this systematic review is presented in detail and all steps of the process are accounted for. In this section, we go from the definition of the research questions to the selection of the works to be included in this review, while explicating the inclusion and exclusion criteria and the data sources considered.
	\item Section \ref{sec:res}, in which we show which answers to the different research questions are proposed by the analysed studies to the best of our knowledge.
	\item Section \ref{sec:disc}, in which we take a deeper look at these studies and identify potential challenges and future directions for research.
	\item Section \ref{sec:conc}, which outlines the Conclusions and Future Work.
\end{itemize}


\section{Method}
\label{sec:meth}

\subsection{Research Questions}

\subsection{Data Sources}

\subsection{Search Terms}

\subsection{Quality assessment: inclusion and exclusion criteria}

\subsection{Data Extraction}

\section{Results}
\label{sec:res}

\subsection{RQX}

\section{Discussion}
\label{sec:disc}

\subsection{RQX}

\section{Conclusion and future work}
\label{sec:conc}

\bibliographystyle{IEEEtran}
\bibliography{refs}

\end{document}